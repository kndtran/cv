%%%%%%%%%%%%%%%%%%%%%%%%%%%%%%%%%%%%%%%%%%%%%%%%%%%%%%%%%%%%%%%%%%%%%%%%
%%%%%%%%%%%%%%%%%%%%%% Simple LaTeX CV Template %%%%%%%%%%%%%%%%%%%%%%%%
%%%%%%%%%%%%%%%%%%%%%%%%%%%%%%%%%%%%%%%%%%%%%%%%%%%%%%%%%%%%%%%%%%%%%%%%

%%%%%%%%%%%%%%%%%%%%%%%%%%%%%%%%%%%%%%%%%%%%%%%%%%%%%%%%%%%%%%%%%%%%%%%%
%% NOTE: If you find that it says                                     %%
%%                                                                    %%
%%                           1 of ??                                  %%
%%                                                                    %%
%% at the bottom of your first page, this means that the AUX file     %%
%% was not available when you ran LaTeX on this source. Simply RERUN  %%
%% LaTeX to get the ``??'' replaced with the number of the last page  %%
%% of the document. The AUX file will be generated on the first run   %%
%% of LaTeX and used on the second run to fill in all of the          %%
%% references.                                                        %%
%%%%%%%%%%%%%%%%%%%%%%%%%%%%%%%%%%%%%%%%%%%%%%%%%%%%%%%%%%%%%%%%%%%%%%%%

%%%%%%%%%%%%%%%%%%%%%%%%%%%%%%%%%%%%%%%%%%%%%%%%%%%%%%%%%%%%%%%%%%%%%%%%
%% NOTE: If you are getting compilation errors referring to list      %%
%%       definitions that don't match, you may need to upgrade to a   %%
%%       newer version of the enumitem package. Try going to:         %%
%%                                                                    %%
%%   http://www.ctan.org/tex-archive/macros/latex/contrib/enumitem    %%
%%                                                                    %%
%%       then download the enumitem.sty file from there. Place it in  %%
%%       the same directory as your CV. So long as there are no other %%
%%       conflicts with older packages on your system, hopefully that %%
%%       will fix your compilation problems.                          %%
%%%%%%%%%%%%%%%%%%%%%%%%%%%%%%%%%%%%%%%%%%%%%%%%%%%%%%%%%%%%%%%%%%%%%%%%

%%%%%%%%%%%%%%%%%%%%%%%%%%%% Document Setup %%%%%%%%%%%%%%%%%%%%%%%%%%%%

% Don't like 10pt? Try 11pt or 12pt
\documentclass[10pt]{article}

\usepackage[bookmarks=false]{hyperref}

% The automated optical recognition software used to digitize resume
% information works best with fonts that do not have serifs. This
% command uses a sans serif font throughout. Uncomment both lines (or at
% least the second) to restore a Roman font (i.e., a font with serifs).
%\usepackage{times}
%\renewcommand{\familydefault}{\sfdefault}

\usepackage{comment}
\usepackage{pdfpages}
\usepackage[utf8]{inputenc}
%\usepackage[T1]{fontenc}

% This is a helpful package that puts math inside length specifications
\usepackage{calc}

% Layout: Puts the section titles on left side of page
\reversemarginpar

%
%         PAPER SIZE, PAGE NUMBER, AND DOCUMENT LAYOUT NOTES:
%
% The next \usepackage line changes the layout for CV style section
% headings as marginal notes. It also sets up the paper size as either
% letter or A4. By default, letter was used. If A4 paper is desired,
% comment out the letterpaper lines and uncomment the a4paper lines.
%
% As you can see, the margin widths and section title widths can be
% easily adjusted.
%
% ALSO: Notice that the includefoot option can be commented OUT in order
% to put the PAGE NUMBER *IN* the bottom margin. This will make the
% effective text area larger.
%
% IF YOU WISH TO REMOVE THE ``of LASTPAGE'' next to each page number,
% see the note about the +LP and -LP lines below. Comment out the +LP
% and uncomment the -LP.
%
% IF YOU WISH TO REMOVE PAGE NUMBERS, be sure that the includefoot line
% is uncommented and ALSO uncomment the \pagestyle{empty} a few lines
% below.
%

%% Use these lines for letter-sized paper
\usepackage[paper=letterpaper,
            %includefoot, % Uncomment to put page number above margin
            marginparwidth=1.2in,     % Length of section titles
            marginparsep=.05in,       % Space between titles and text
            margin=1in,               % 1 inch margins
            includemp]{geometry}

%% Use these lines for A4-sized paper
%\usepackage[paper=a4paper,
%            %includefoot, % Uncomment to put page number above margin
%            marginparwidth=30.5mm,    % Length of section titles
%            marginparsep=1.5mm,       % Space between titles and text
%            margin=25mm,              % 25mm margins
%            includemp]{geometry}

%% More layout: Get rid of indenting throughout entire document
\setlength{\parindent}{0in}

\usepackage[shortlabels]{enumitem}

% Simpler bibsections for CV sections
% (thanks to natbib for inspiration)
%
% * For lists of references with hanging indents and no numbers:
%
%   \begin{bibsection}
%       \item ...
%   \end{bibsection}
%
% * For numbered lists of references (with hanging indents):
%
%   \begin{bibenum}
%       \item ...
%   \end{bibenum}
%
%   Note that bibenum numbers continuously throughout. To reset the
%   counter, use
%
%   \restartlist{bibenum}
%
%   at the place where you want the numbering to reset.

\makeatletter
\newlength{\bibhang}
\setlength{\bibhang}{1em}
\newlength{\bibsep}
 {\@listi \global\bibsep\itemsep \global\advance\bibsep by\parsep}
\newlist{bibsection}{itemize}{3}
\setlist[bibsection]{label=,leftmargin=\bibhang}
\newlist{bibenum}{enumerate}{3}
\setlist[bibenum]{resume,label=[\arabic*],leftmargin={\bibhang+\widthof{[999]}}}
\setlist*[bibsection,bibenum]{%
        itemindent=-\bibhang,
        itemsep=\bibsep,parsep=\z@,partopsep=0pt,
        topsep=0pt}
\let\oldendbibenum\endbibenum
\def\endbibenum{\oldendbibenum\vspace{-.6\baselineskip}}
\let\oldendbibsection\endbibsection
\def\endbibsection{\oldendbibsection\vspace{-.6\baselineskip}}
\makeatother

%% Reference the last page in the page number
%
% NOTE: comment the +LP line and uncomment the -LP line to have page
%       numbers without the ``of ##'' last page reference)
%
% NOTE: uncomment the \pagestyle{empty} line to get rid of all page
%       numbers (make sure includefoot is commented out above)
%
\usepackage{fancyhdr,lastpage}
\pagestyle{fancy}
%\pagestyle{empty}      % Uncomment this to get rid of page numbers
\fancyhf{}\renewcommand{\headrulewidth}{0pt}
\fancyfootoffset{\marginparsep+\marginparwidth}
\newlength{\footpageshift}
\setlength{\footpageshift}
          {0.5\textwidth+0.5\marginparsep+0.5\marginparwidth-2in}
\lfoot{\hspace{\footpageshift}%
       \parbox{4in}{\, \hfill %
                    \arabic{page} of \protect\pageref*{LastPage} % +LP
%                    \arabic{page}                               % -LP
                    \hfill \,}}

% Finally, give us PDF bookmarks
\usepackage{color,hyperref}
\definecolor{darkblue}{rgb}{0.0,0.0,0.3}
\hypersetup{colorlinks,breaklinks,
            linkcolor=darkblue,urlcolor=darkblue,
            anchorcolor=darkblue,citecolor=darkblue}

%%%%%%%%%%%%%%%%%%%%%%%% End Document Setup %%%%%%%%%%%%%%%%%%%%%%%%%%%%


%%%%%%%%%%%%%%%%%%%%%%%%%%% Helper Commands %%%%%%%%%%%%%%%%%%%%%%%%%%%%

%%% HEADING AT TOP OF CURRICULUM VITAE

% The title (name) with a horizontal rule under it
% (optional argument typesets an object right-justified across from name
%  as well)
%
% Usage: \makeheading{name}
%        OR
%        \makeheading[right_object]{name}
%
% Place at top of document. It should be the first thing.
% If ``right_object'' is provided in the square-braced optional
% argument, it will be right justified on the same line as ``name'' at
% the top of the CV. For example:
%
%       \makeheading[\emph{Curriculum vitae}]{Your Name}
%
% will put an emphasized ``Curriculum vitae'' at the top of the document
% as a title. Likewise, a picture could be included:
%
%   \makeheading[\includegraphics[height=1.5in]{my_picutre}]{Your Name}
%
% the picture will be flush right across from the name.
\newcommand{\makeheading}[2][]%
        {\hspace*{-\marginparsep minus \marginparwidth}%
         \begin{minipage}[t]{\textwidth+\marginparwidth+\marginparsep}%
             {\large \bfseries #2 \hfill #1}\\[-0.15\baselineskip]%
                 \rule{\columnwidth}{1pt}%
         \end{minipage}}

%%% SECTION HEADINGS

% The section headings. Flush left in small caps down pseudo-margin.
%
% Usage: \section{section name}
\renewcommand{\section}[1]{\pagebreak[3]%
    \hyphenpenalty=10000%
    \vspace{1.3\baselineskip}%
    \phantomsection\addcontentsline{toc}{section}{#1}%
    \noindent\llap{\scshape\smash{\parbox[t]{\marginparwidth}{\raggedright #1}}}%
    \vspace{-\baselineskip}\par}

%%% LISTS

% This macro alters a list by removing some of the space that follows the list
% (is used by lists below)
\newcommand*\fixendlist[1]{%
    \expandafter\let\csname preFixEndListend#1\expandafter\endcsname\csname end#1\endcsname
    \expandafter\def\csname end#1\endcsname{\csname preFixEndListend#1\endcsname\vspace{-0.6\baselineskip}}}

% These macros help ensure that items in outer-type lists do not get
% separated from the next line by a page break
% (they are used by lists below)
\let\originalItem\item
\newcommand*\fixouterlist[1]{%
    \expandafter\let\csname preFixOuterList#1\expandafter\endcsname\csname #1\endcsname
    \expandafter\def\csname #1\endcsname{\csname preFixOuterList#1\endcsname\let\oldItem\item\def\item{\pagebreak[2]\oldItem}}
    \expandafter\let\csname preFixOuterListend#1\expandafter\endcsname\csname end#1\endcsname
    \expandafter\def\csname end#1\endcsname{\let\item\oldItem\csname preFixOuterListend#1\endcsname}}
\newcommand*\fixinnerlist[1]{%
    \expandafter\let\csname preFixInnerList#1\expandafter\endcsname\csname #1\endcsname
    \expandafter\def\csname #1\endcsname{\let\oldItem\item\let\item\originalItem\csname preFixInnerList#1\endcsname}
    \expandafter\let\csname preFixInnerListend#1\expandafter\endcsname\csname end#1\endcsname
    \expandafter\def\csname end#1\endcsname{\csname preFixInnerListend#1\endcsname\let\item\oldItem}}

% An itemize-style list with lots of space between items
%
% Usage:
%   \begin{outerlist}
%       \item ...    % (or \item[] for no bullet)
%   \end{outerlist}
\newlist{outerlist}{itemize}{3}
    \setlist[outerlist]{label=\enskip\textbullet,leftmargin=*}
    \fixendlist{outerlist}
    \fixouterlist{outerlist}

% An environment IDENTICAL to outerlist that has better pre-list spacing
% when used as the first thing in a \section
%
% Usage:
%   \begin{lonelist}
%       \item ...    % (or \item[] for no bullet)
%   \end{lonelist}
\newlist{lonelist}{itemize}{3}
    \setlist[lonelist]{label=\enskip\textbullet,leftmargin=*,partopsep=0pt,topsep=0pt}
    \fixendlist{lonelist}
    \fixouterlist{lonelist}

% An itemize-style list with little space between items
%
% Usage:
%   \begin{innerlist}
%       \item ...    % (or \item[] for no bullet)
%   \end{innerlist}
\newlist{innerlist}{itemize}{3}
    \setlist[innerlist]{label=\enskip\textbullet,leftmargin=*,parsep=0pt,itemsep=0pt,topsep=0pt,partopsep=0pt}
    \fixinnerlist{innerlist}

% An environment IDENTICAL to innerlist that has better pre-list spacing
% when used as the first thing in a \section
%
% Usage:
%   \begin{loneinnerlist}
%       \item ...    % (or \item[] for no bullet)
%   \end{loneinnerlist}
\newlist{loneinnerlist}{itemize}{3}
    \setlist[loneinnerlist]{label=\enskip\textbullet,leftmargin=*,parsep=0pt,itemsep=0pt,topsep=0pt,partopsep=0pt}
    \fixendlist{loneinnerlist}
    \fixinnerlist{loneinnerlist}

%%% EXTRA SPACE

% To add some paragraph space between lines.
% This also tells LaTeX to preferably break a page on one of these gaps
% if there is a needed pagebreak nearby.
\newcommand{\blankline}{\quad\pagebreak[3]}
\newcommand{\halfblankline}{\quad\vspace{-0.5\baselineskip}\pagebreak[3]}

%%% FORMATTING MACROS

% Uses hyperref to link DOI
\newcommand\doilink[1]{\href{http://dx.doi.org/#1}{#1}}
\newcommand\doi[1]{doi:\doilink{#1}}

% For \url{SOME_URL}, links SOME_URL to the url SOME_URL
\providecommand*\url[1]{\href{#1}{#1}}
% Same as above, but pretty-prints SOME_URL in teletype fixed-width font
\renewcommand*\url[1]{\href{#1}{\texttt{#1}}}

% For \email{ADDRESS}, links ADDRESS to the url mailto:ADDRESS
\providecommand*\email[1]{\href{mailto:#1}{#1}}
% Same as above, but pretty-prints ADDRESS in teletype fixed-width font
%\renewcommand*\email[1]{\href{mailto:#1}{\texttt{#1}}}

%\providecommand\BibTeX{{\rm B\kern-.05em{\sc i\kern-.025em b}\kern-.08em
%    T\kern-.1667em\lower.7ex\hbox{E}\kern-.125emX}}
%\providecommand\BibTeX{{\rm B\kern-.05em{\sc i\kern-.025em b}\kern-.08em
%    \TeX}}
\providecommand\BibTeX{{B\kern-.05em{\sc i\kern-.025em b}\kern-.08em
    \TeX}}
\providecommand\Matlab{\textsc{Matlab}}

%%%%%%%%%%%%%%%%%%%%%%%% End Helper Commands %%%%%%%%%%%%%%%%%%%%%%%%%%%

%%%%%%%%%%%%%%%%%%%%%%%%% Begin CV Document %%%%%%%%%%%%%%%%%%%%%%%%%%%%

\begin{document}
\makeheading{Resume: Khoi-Nguyen Tran}

\section{Contact}

% NOTE: Mind where the & separators and \\ breaks are in the following
%       table. Table is one row made up of three parboxes. The left
%       parbox has address info, the middle parbox has a vertical bar,
%       and the right parbox has phone and electronic contact
%       information.
%
% MACROS: \rcollength is the width of the right column of the table
%             (adjust it to your liking; default is 1.85in).
%         \spacewidth is width of area between left and right boxes.
%         \spacechar is character used to produce perforated vertical
%             boundary between boxes.
%
\newlength{\rcollength}\setlength{\rcollength}{2.00in}%
\newlength{\spacewidth}\setlength{\spacewidth}{20pt}
\newcommand\spacechar{$ $} %|
%
\begin{tabular}[t]{@{}p{\textwidth-\rcollength-\spacewidth}@{}p{\spacewidth}@{}p{\rcollength}}%

%%%%%%%%%%%%%%%%%%%%%%%%%%%%%%%%%%%%%%%%%%%%%%%%%%%%%%%%%%%%%%%%%%%%%%%%%%%%%%%
%%%%%%%%%%%%%%%%%%%%%%%%%%%%%%%%%%%%%%%%%%%%%%%%%%%%%%%%%%%%%%%%%%%%%%%%%%%%%%%
%%%%%%%%%%%%%%%%%%%%%%%%%%%%%%%%%%%%%%%%%%%%%%%%%%%%%%%%%%%%%%%%%%%%%%%%%%%%%%%

% Address box
\parbox{\textwidth-\rcollength-\spacewidth}{%
IBM Research\\
60 City Road,\\
Southbank, VIC 3006,\\
Australia}

% Cheesy perforated vertical bar between boxes
% Shorten by removing \spacechar's
& \parbox{\spacewidth}{\centering \spacechar\\\spacechar\\\spacechar\\\spacechar\\\spacechar} &

% Non-snail-mail contact information
\parbox{\rcollength}{%
\textit{Mobile:} +61 424 316 544\\
\textit{Email:} \email{kndtran@gmail.com}\\
\textit{WWW:} \href{http://kndtran.com/}{kndtran.com}}

\end{tabular}

%%%%%%%%%%%%%%%%%%%%%%%%%%%%%%%%%%%%%%%%%%%%%%%%%%%%%%%%%%%%%%%%%%%%%%%%%%%%%%%
%%%%%%%%%%%%%%%%%%%%%%%%%%%%%%%%%%%%%%%%%%%%%%%%%%%%%%%%%%%%%%%%%%%%%%%%%%%%%%%
%%%%%%%%%%%%%%%%%%%%%%%%%%%%%%%%%%%%%%%%%%%%%%%%%%%%%%%%%%%%%%%%%%%%%%%%%%%%%%%

%\section{Citizenship}

%Australian

%%
%% In modern CV's, it seems like ``Objective'' is frowned upon. Instead,
%% incorporate it into a well-constructed cover letter. The ``More
%% information'' can go at the end of the CV, but it should not distract
%% from the section giving references available to contact.
%%
%
% \section{Objective}
%
% Placement in an academic position (i.e., faculty, postdoctoral, or
% research scientist) that allows for advanced research in distributed
% complex adaptive systems (i.e., modeling, analysis, design, and
% verification) with a particular focus on the control of engineered
% agents (e.g., for communications, control, software, electronics, and
% sustainability) and the analysis of biological phenomena (e.g.,
% self-organization, ecological rationality)
% \begin{innerlist}
% \item More information and auxiliary documents can be found at\\\url{http://www.tedpavlic.com/facjobsearch/}
% \end{innerlist}

\halfblankline

%%%%%%%%%%%%%%%%%%%%%%%%%%%%%%%%%%%%%%%%%%%%%%%%%%%%%%%%%%%%%%%%%%%%%%%%%%%%%%%
%%%%%%%%%%%%%%%%%%%%%%%%%%%%%%%%%%%%%%%%%%%%%%%%%%%%%%%%%%%%%%%%%%%%%%%%%%%%%%%
%%%%%%%%%%%%%%%%%%%%%%%%%%%%%%%%%%%%%%%%%%%%%%%%%%%%%%%%%%%%%%%%%%%%%%%%%%%%%%%

\section{Employment}

\href{http://www.research.ibm.com/labs/australia/}{\textbf{IBM Research -- Australia}}\\
Melbourne, VIC, Australia

\begin{outerlist}
	\item Postdoctoral Researcher. Cognitive Analytics Group. \hfill {2016 -- Current}
	\begin{innerlist}
		\item[$-$] Led development of research components on understanding text in PDF documents for chunking and learning objective generation.
		\begin{innerlist}
			\item[$*$] Delivered 3 web based apps/demos in the IBM Cloud.
			% to the client and software teams, for use by the sales team, data science team, and SME team.
			\item[$*$] Delivered and maintained Java and Python software packages for production system integration.
			% for the software engineering team's integration into the production system.
			\item[$*$] Research team was recognised as providing leading expertise on development.
			\item[$*$] 1 research publication under review.
		%\item[$-$] Developing an end-to-end research based classification system for detecting vandalism on Wikipedia, incorporating traditional feature engineering techniques and exploring deep learning techniques.
		\end{innerlist}
		\item[$-$] Exploring machine/deep learning applications for the Education Library, focusing on document classification and alignment of text content to learning standards.
		\begin{innerlist}
			\item[$*$] Delivered 2 Java software packages, wrapping past research code for production.
			% level system with parallelism and simplified execution.
			\item[$*$] Developing one Python web service to improve 
			%using supervised learning to improve the search in the unsupervised learning algorithm.
		\end{innerlist}			
\end{innerlist}
\end{outerlist}

\halfblankline

\href{http://www.australia.gov.au/}{\textbf{Australian Federal Government}}\\
Melbourne, VIC, Australia

\begin{outerlist}
	\item Data Scientist. Research and Development Team. \hfill {2015 -- 2016}
	%\begin{innerlist}
		%\item[$-$] Led development of advanced profiling techniques in collaboration with senior data analysts and data engineers.
		%\item[$-$] Revised production profiling scripts for correctness and reimplementation into a new production system.
		%\item[$-$] Lead writer and investigator of one of four reports into the data quality issues in production databases.
  %\end{innerlist}
\end{outerlist}

\halfblankline

\href{http://www.anu.edu.au/}{\textbf{The Australian National University (ANU)}}\\
Canberra, ACT, Australia

\begin{outerlist}
	\item Co-founder and Research Assistant. \href{http://sociology.cass.anu.edu.au/centres/anu-cybercrime}{ANU Cybercrime Observatory.} \hfill {2013 -- 2015}
	\begin{innerlist}
		%\item[$-$] Project: \textit{Tracking Cybercrime Activities in Australia}
		\item[$-$] Led research into detecting spam emails with malicious attachments or URLs.
		\item[$-$] First author full paper publication in AusDM (2013).
		%\item[$-$] Provided technical expertise on generating statistics and deep data analysis of multiple spam email data sources.
		%\item[$-$] Managed data security and computer systems for research.
		%\item[$-$] Significant contributions to an ethics proposal for a research project on people's susceptibility to cybercrime.
		%\item[$-$] Developed novel detection technique for malicious content based on email content.
		%\item[$-$] Applied technique and other analyses to over 13 million spam emails, consisting of over 1 million attachments and over 21 million URLs.
		%\item[$-$] Full paper publication in AusDM (2013).
    \end{innerlist}
	
	\halfblankline
	
	\item Research Assistant. \href{http://cs.anu.edu.au/}{Research School of Computer Science.} \hfill {2011}
	\begin{innerlist}
		%\item[$-$] Project: \textit{Preference Elicitation for Social Recommendation}
		\item[$-$] Led development of a Facebook app (\href{http://linkr.anu.edu.au}{http://linkr.anu.edu.au}) to collect user data for research into collaborative filtering for recommendation tasks.
		\item[$-$] Co-authored full paper publication in WWW (2012).
		%\item[$-$] Recruited and supported over 200 participants in a 5 month user study.
		%\item[$-$] Collected usage data from over 37,000 Facebook users.
		%\item[$-$] Co-authored and presented full paper research publication at the WWW 2012 conference.
	\end{innerlist}
\end{outerlist}

%%%%%%%%%%%%%%%%%%%%%%%%%%%%%%%%%%%%%%%%%%%%%%%%%%%%%%%%%%%%%%%%%%%%%%%%%%%%%%%
%%%%%%%%%%%%%%%%%%%%%%%%%%%%%%%%%%%%%%%%%%%%%%%%%%%%%%%%%%%%%%%%%%%%%%%%%%%%%%%
%%%%%%%%%%%%%%%%%%%%%%%%%%%%%%%%%%%%%%%%%%%%%%%%%%%%%%%%%%%%%%%%%%%%%%%%%%%%%%%

%\section{Programming}

%Languages:
%
%\begin{innerlist}
%    \item Experienced: R, Python, SQL, UNIX shell scripting
%	\item Familiar: Javascript, PHP, Haskell, Java, C, \Matlab
%\end{innerlist}

% \halfblankline

% Software:
% %
% \begin{innerlist}
    % \item Experienced: dplyr, tidyr, Stanford CoreNLP, Scikit-learn (Python machine learning library), \LaTeX , JQuery, MySQL
	% \item Familiar: Eclipse, Apache, SVN/Git, Ubuntu Linux, Microsoft Windows XP/7/8, Microsoft Word/Excel/Powerpoint
% \end{innerlist}

%%%%%%%%%%%%%%%%%%%%%%%%%%%%%%%%%%%%%%%%%%%%%%%%%%%%%%%%%%%%%%%%%%%%%%%%%%%%%%%
%%%%%%%%%%%%%%%%%%%%%%%%%%%%%%%%%%%%%%%%%%%%%%%%%%%%%%%%%%%%%%%%%%%%%%%%%%%%%%%
%%%%%%%%%%%%%%%%%%%%%%%%%%%%%%%%%%%%%%%%%%%%%%%%%%%%%%%%%%%%%%%%%%%%%%%%%%%%%%%

\section{Education}

\href{http://www.anu.edu.au/}{\textbf{The Australian National University (ANU)}}\\
Canberra, ACT, Australia

\begin{outerlist}
\item Ph.D. in Computer Science. \hfill 2015
	\begin{innerlist}
		%\item Thesis Topic: \emph{Design and Analysis of Optimal Task-Processing Agents}
		\item[$-$] Area of Study: Machine learning and its applications
		\item[$-$] Thesis Title: \emph{Detecting Vandalism on Wikipedia across Multiple Languages}
		\item[$-$] First author full paper publications in TKDE (2015), PAKDD (2013, 2015 - Best Student Paper), CIKM (2013 - Poster and Demo), and AusDM (2013).
		%\item[$-$] Developed novel vandalism detection methods based on metadata and text data.
		%\item[$-$] Demonstrated effective reuse of machine learning models across multiple languages.
	\end{innerlist}

\item Bachelor of Computer Science, with First Class Honours. \hfill 2009
\end{outerlist}

%%%%%%%%%%%%%%%%%%%%%%%%%%%%%%%%%%%%%%%%%%%%%%%%%%%%%%%%%%%%%%%%%%%%%%%%%%%%%%%
%%%%%%%%%%%%%%%%%%%%%%%%%%%%%%%%%%%%%%%%%%%%%%%%%%%%%%%%%%%%%%%%%%%%%%%%%%%%%%%
%%%%%%%%%%%%%%%%%%%%%%%%%%%%%%%%%%%%%%%%%%%%%%%%%%%%%%%%%%%%%%%%%%%%%%%%%%%%%%%
\begin{comment}
\section{Service}

\href{http://cs.club.anu.edu.au}{\textbf{ANU Computer Science Students' Association}}
\begin{innerlist}
    \item Volunteer, 2007--2010
	\item President, 2011--2012
	\begin{innerlist}
		\item[$-$] Total membership for 2011: 220+, for 2012: 300+.
		\item[$-$] Notable events: 20+ BBQs, 4 Epic Games Nights, 8 technical talks (incl. Kinect development workshop), 4 software installation events, and 2 membership recruitments.
		\item[$-$] Winner of ``Event of the Year'' in 2012 for the Epic Games Night.
		\item[$-$] Runner up for ``Club of the Year'' in 2012.
	\end{innerlist}
    \item \href{http://www.microsoftstudentpartners.com/}{Microsoft Student Ambassador}, 2011--2012
\end{innerlist}
\end{comment}
%%%%%%%%%%%%%%%%%%%%%%%%%%%%%%%%%%%%%%%%%%%%%%%%%%%%%%%%%%%%%%%%%%%%%%%%%%%%%%%
%%%%%%%%%%%%%%%%%%%%%%%%%%%%%%%%%%%%%%%%%%%%%%%%%%%%%%%%%%%%%%%%%%%%%%%%%%%%%%%
%%%%%%%%%%%%%%%%%%%%%%%%%%%%%%%%%%%%%%%%%%%%%%%%%%%%%%%%%%%%%%%%%%%%%%%%%%%%%%%

%\section{General Interests}
%
%\begin{innerlist}
%	\item Computer games, card games, board games
%	\item Cooking, baking bread, growing food, kickboxing, badminton
%	\item Learning human languages (French, Mandarin Chinese, Spanish, Russian, and German)
%\end{innerlist}

%%%%%%%%%%%%%%%%%%%%%%%%%%%%%%%%%%%%%%%%%%%%%%%%%%%%%%%%%%%%%%%%%%%%%%%%%%%%%%%
%%%%%%%%%%%%%%%%%%%%%%%%%%%%%%%%%%%%%%%%%%%%%%%%%%%%%%%%%%%%%%%%%%%%%%%%%%%%%%%
%%%%%%%%%%%%%%%%%%%%%%%%%%%%%%%%%%%%%%%%%%%%%%%%%%%%%%%%%%%%%%%%%%%%%%%%%%%%%%%

\section{References}

Available on request.

% \href
% {http://cs.anu.edu.au/people/Peter.Christen/}
% {\textbf{Prof.~Peter~Christen}}
% (email:~\href{mailto:peter.christen@anu.edu.au}{peter.christen@anu.edu.au})
% %
% \begin{innerlist}
	% \item Associate Professor,
		% \href{http://cs.anu.edu.au/}{Research School of Computer Science}
	% \item The Australian National University, Canberra, ACT 2601, Australia
    % \item \emph{Prof.~Christen is my graduate supervisor and panel chair for my PhD candidacy.}
% \end{innerlist}

% \halfblankline

% \href
% {http://regnet.anu.edu.au/people/professor-roderic-broadhurst}
% {\textbf{Prof.~Roderic~Broadhurst}}
% (email:~\href{mailto:roderic.broadhurst@anu.edu.au}{roderic.broadhurst@anu.edu.au})
% %
% \begin{innerlist}
	% \item Professor,
		% \href{http://cass.anu.edu.au/}{College of Arts and Social Science}
	% \item The Australian National University, Canberra, ACT 2601, Australia
    % \item \emph{Prof.~Broadhurst is my supervisor at the ANU Cybercrime Observatory.}
% \end{innerlist}





%\halfblankline

% \href
% {http://regnet.anu.edu.au/people/dr-mamoun-alazab}
% {\textbf{Dr.~Mamoun~Alazab}}
% (email:~\href{mailto:mamoun.alazab@anu.edu.au}{mamoun.alazab@anu.edu.au})
% %
% \begin{innerlist}
	% \item Research Associate,
		% \href{http://cass.anu.edu.au/}{College of Arts and Social Science}
	% \item The Australian National University, Canberra, ACT 2601, Australia
    % \item \emph{Dr.~Alazab was my main supervisor for the project named ``Tracking Cybercrime Activities in Australia''.}
% \end{innerlist}

%\item[$\diamond$]
\end{document}

%%%%%%%%%%%%%%%%%%%%%%%%%% End CV Document %%%%%%%%%%%%%%%%%%%%%%%%%%%%%

%----------------------------------------------------------------------%
% The following is copyright and licensing information for
% redistribution of this LaTeX source code; it also includes a liability
% statement. If this source code is not being redistributed to others,
% it may be omitted. It has no effect on the function of the above code.
%----------------------------------------------------------------------%
% Copyright (c) 2007, 2008, 2009, 2010, 2011 by Theodore P. Pavlic
%
% Unless otherwise expressly stated, this work is licensed under the
% Creative Commons Attribution-Noncommercial 3.0 United States License. To
% view a copy of this license, visit
% http://creativecommons.org/licenses/by-nc/3.0/us/ or send a letter to
% Creative Commons, 171 Second Street, Suite 300, San Francisco,
% California, 94105, USA.
%
% THE SOFTWARE IS PROVIDED "AS IS", WITHOUT WARRANTY OF ANY KIND, EXPRESS
% OR IMPLIED, INCLUDING BUT NOT LIMITED TO THE WARRANTIES OF
% MERCHANTABILITY, FITNESS FOR A PARTICULAR PURPOSE AND NONINFRINGEMENT.
% IN NO EVENT SHALL THE AUTHORS OR COPYRIGHT HOLDERS BE LIABLE FOR ANY
% CLAIM, DAMAGES OR OTHER LIABILITY, WHETHER IN AN ACTION OF CONTRACT,
% TORT OR OTHERWISE, ARISING FROM, OUT OF OR IN CONNECTION WITH THE
% SOFTWARE OR THE USE OR OTHER DEALINGS IN THE SOFTWARE.
%----------------------------------------------------------------------%
